\documentclass[11pt,a4paper]{article}
\textheight245mm
\textwidth170mm
\hoffset-21mm
\voffset-15mm
\parindent0pt
\usepackage[utf8]{inputenc}
\usepackage[english]{babel}
\usepackage{amsmath,amsfonts,amssymb}
\usepackage{amsmath}
\usepackage{graphicx}
\usepackage{caption}
\usepackage{fancyhdr}
\pagestyle{fancy}

\renewcommand{\headrulewidth}{1pt}
\fancyhead[C]{Rapport}
\fancyhead[L]{L3 Informatique - 2022/2023}
\fancyhead[R]{Projet Programmation}

\renewcommand{\footrulewidth}{1pt}
\fancyfoot[C]{\thepage} 
\fancyfoot[L]{Sacha Ben-Arous}
\fancyfoot[R]{E.N.S Paris-Saclay}

\begin{document}
\section{Problèmes rencontrés :}
 Le premier problème que j'ai rencontré est celui de la division euclidienne en assembleur : en lisant la doc, on apprend que deux registres sont concaténés pour former la dividende, et le raisonnement naïf que j'ai eu consiste à mettre la dividende dans un registre, et mettre 0 dans l'autre, de manière à ne pas se soucier de la concaténation. Cependant, dans le cas d'entiers négatifs, cela pose problème car le bit de signe de l'entier est oublié, et devient un bit quelconque qui modifie la valeur de cet entier. La solution que j'ai choisie d'utiliser l'instruction "cqto" , qui permet de correctement répartir la dividende entre les deux registres. J'en ai par ailleurs profité pour modifier le comportement du modulo, de tel sorte que le reste renvoyé soit toujours entre le diviseur et 0 : par exemple, sans modification $ -2 \% 3 $ donnait en retour $-2$, alors que avec mon implémentation, le résultats est $1$. De même $2 \%-3$ qui donnait $2$ renvoie maintenant $-1$. \\ \\
Le second problème est survenu lors de la gestion des variables : lors de certains cas précis (i.e : aléatoires) de changement de type, des résultats complètement incohérents étaient obtenus : par exemple "$x=1 \ ; y=2 \ ; x=3.1 \ ; \ y$" avait comme valeur de retour $0$. La solution que j'ai choisie consiste à initialiser toutes les variables comme des ".double", même dans le cas d'entiers.

\section{Implémentation des bonus :}
J'ai implémenté la factorielle et la puissance. Ces opérateurs prennent des entiers en entrée et renvoient des entiers. Le calcul de "$a$ factorielle" s'écrit "$a!$" , et vaut $1$ si $a \leq -1$, et coïncide avec la factorielle usuelle sinon. Ensuite, j'ai choisi d'implementer le calcul de "$a$ puissance $b$" par "$a** \ b$", avec $a$ et $b$ des entiers. Si $b$ est négatif, le résultat sera toujours $1$, et sinon il correspond à l'exponentiation usuelle. Les priorités de ces opérateurs sont celles classiquement attribuées.\\ \\
J'ai aussi implémenté la gestion des variables, et par la même occasion la gestion de plusieurs lignes de calcul. Une ligne est soit une définition de variable, qui utilise les opérateurs implémentés, et éventuellement des variables précédemment définies ; soit un calcul dont le résultat sera renvoyé. De plus, une variable peut être redéfinie, et ce même en fonction d'elle-même. Par exemple $a=1 \ ; a=a+1 \ ; a \ $ aura comme valeur retournée $2$. Il faut que le fichier expression termine par au moins une ligne de calcul pour être accepté. Voir les exemples pour plus de détails.


\end{document}
